Teil D beschäftigt sich mit der praktischen Relevanz der vergangenen Teile.
Teil A gibt eine Einführung in Entscheidungen, Teil B beschäftigt sich mit
der präskriptiven Entscheidungstheorie und Teil C gibt einen Einblick in
die menschlichen Verhaltensweisen und Irrationalitäten.

\subsection{Anwendungsfelder}
\begin{itemize}
    \item Verbesserung der Entscheidungsqualität - Einfluss auf den
        rational-analytischen Entscheidungsprozess, Eliminierung von
        Verzerrungen durch psychologische Effekte
    \item Beeinflussung des Verhaltens Dritter zum eigenen Nutzen - z.B.
        Unternehmen, die geziehlt Lockangebote und gute Werbung platzieren
    \item Beeinflussung des Verhaltens Dritter zum gesellschaftlichen Nutzen
        oder zum Nutzer der jeweiligen Person (\textbf{Nudging}) - Das Anstoßen
        kluger Entscheidungen, \textbf{Entscheidungsarchitekten} konstruieren
        gezielt Entscheidungssituationen (z.B. gesundes Obst präsentieren
        anstatt ungesunder Lebensmittel)
    \item Beeinflussung des eigenen Verhaltens -
        Nudging auf die eigene Person anwenden
    \item Veränderung der Wahrnehmung zur Zufriedenheitssteigerung
        - Zusammenfassen positiver und negativer Ereignisse auf mentalem Konto.
        Anstatt 1000EUR Verlust und 1500EUR Gewinn, wird beides auf einem
        mentalen Konto Kummulierter Umsatz zugeschrieben und der
        Entscheider kann sich über 500EUR Gewinn freuen.
    \item Erlangen eines eigenen Profits aus der Verhaltensprognose anderer -
        Vorhersage von Verhalten um Kurse vorherzusagen (schwierig) oder
        Konsumverhalten eines Individuums vorhersagen um passende Angebote
        machen zu können (Facebook und co.)
\end{itemize}
Mit dem Vorwissen der vorhergegangen Teile können verschiedene Effekte nun
ausgenutzt bzw. einbezogen werden:

\begin{itemize}
    \item Kontraksteffekte - hervorheben durch Kontraste oder gezielte Wahl
        von Kontrasten um wichtige Informationen hervorzuheben.
    \item Verfügbarkeitseffekte - besonderes hervorheben von Nachteilen um
        einen Wandel im Handeln einer Person zu erwirken oder als Möglichkeit
        Werbung zu machen. So wird ein Produkt besonders verfügbar gemacht und
        eine positive Stimmung umher erzeugt (z.B. durch Begrüßung und
        narrative Bias).
    \item Verankerungseffekte und Status Quo Bias
        - zu hohen Anker als Preis ansetzen da eine Preissteigerung nur schwierig
        möglich ist. Für einen positiven Einfluss kann der status quo bias
        genutzt werden indem bei Organspenden anstatt einer Zustimmung, die
        Ablehnung einer automatischen Organspende durchgesetzt wird.
    \item Mental Accounting
        - Verluste zusammen verbuchen da einmal steiler Verlust leichter
        zu verkraften ist (Integration). Bei Gewinnen umgekehrt (Segregation).
        Viele Gewinner sind dem Selbstwert dienlicher.
    \item Beeinflussung von Bezugspunkten
    \begin{itemize}
        \item positives und negatives Framing - 1\% sterben vs 99\% überleben.
            Ersteres negativ da deutlich über 0, letzteres positiv da quasi schon 100.
        \item irrelevante Alternativen zur Aufwertung des Angebots präsentieren.
            - In der Hoffnung den Bezugspunkt zu verschieben
    \end{itemize}
\end{itemize}

\subsection{Teamentscheidungen}
Um zu sauberen Teamentscheidung zu gelangen, müssen verschiedene Faktoren beachtet
werden. Ein wichtiger Begriff ist ``Stakeholder''. Ein Steakholder ist eine
Person oder eine Gruppe, die ein berechtigtes Interesse am Verlauf
oder Ergebnis eines Prozesses oder Projektes hat.

\begin{itemize}
    \item \textbf{Theoretische Vorüberlegungen}
    \begin{itemize}
        \item Beziehungen der Stakeholder, Weisungsbefugnisse, Einbeziehung
            der Ziele untergestellter Teams und Personen, Altruismus bei
            Einbeziehung von Fremdzielen
        \item \textbf{instrumentelle Verbindungen} - Ein Stakeholder wird
            nur einbezogen weil er einem Instrumentalziel dient. Die Gewichtung
            der Ziele dieses Stakeholders ist eher gering aber das Verhalten
            des Stakeholders hat Einfluss auf den Projekterfolg. Diese Gruppe
            sollte nicht in den Zielkatalog eines Teams aufgenommen werden.
    \end{itemize}
    \item \textbf{praktische Verfahrensüberlegungen}
    \begin{itemize}
        \item theoretische Exaktheit des Entscheidungsprozesses
        \item hohe Partizipation der Stakeholder
        \item niedriger Gesamtaufwand
    \end{itemize}
    Ein Werkzeug zur Planung der Einbindung von Stakeholdern in den
    Entscheidungsprozess bieten Stakeholder-Netzdiagramme. Hier werden
    jeweils instrumentelle Bedeutung und relative Zielgewichtung
    eingetragen. Dadurch ist das Team gezwungen sich ein Überblick über die
    Stakeholder zu verschaffen, Ziele klar zu definieren und damit den Einbezug
    eines Stakeholders in den Entscheidungsprozess abzulesen.
\end{itemize}

Neben dem Umgang mit Stakeholdern, muss es zu einer Behandlung von
Konflikten zwischen Mitgliedern eines Projektteams kommen.
Hier unterscheidet man zwischen \textbf{Meinungsunterschieden}, welche
\textbf{ad hoc} (unbekannte Herkunft) oder \textbf{begründet} (unterschiedliche Vorstellung von Zusammenhängen) sind, und Interessenskonflikten.
Diese sind entweder lösbar durch Kompromisse oder nicht lösbar und es kommt
zu einem Machtspiel.
Bei Meinungsunterschieden kann es auf verschiedenen Wegen zur Auflösung kommen.
Bei ad-hoc Meinungsunterschieden wurde ein Teammitglied von einer Entscheidung
überrumpelt. Hier muss es zur Diskussion der Vor unt Nachteile kommen.
Eine solche Situation wird durch die gemeinsame Entscheidungsfindung
weitgehend eliminiert. Bei begründeten Meinungsunterschieden gibt es mehrere
Möglichkeiten der Ursache:
\begin{itemize}
    \item unzureichend fundamentale Ziele
    \item Bias in Prognosen - z.B. narrative Bias
    \item schlechte Entscheidungsgrundlage
\end{itemize}
Eine Auflösung erfolgt jeweils durch Iteration und Diskussion der Probleme.
Im Falle von Interessenskonflikten kann man verschiedene Gruppentypen
unterscheiden:
\begin{itemize}
    \item \textbf{Kompromiss bei Zielgewichtung} - alle Gruppenmitglieder haben
    fast identische Ziele und gleichzeitig sind die Zielgewichte ähnlich. In diesem
    Fall muss kein Gruppenmitglied weit von den eigenen Vorstellunge abweichen
    und es lässt sich ein Kompromiss finden. Dies kann
    durch die gemeinsame Gewichtung von Zielen erfolgen.
    \item \textbf{Kompromiss bei Auswahl der Alternative} - Die Gruppenmitglieder
    haben deutliche Unterschiede in den Zielgewichten. Eine einfache Gewichtung
    ist nicht mehr möglich. Es muss also ein Kompromiss durch die Suche
    einer Alternative gefunden werden.
    \item \textbf{Machtspiel} - Hier fehlt es an Kooperationsbereitschaft
    und der Interessenkonflikt ist mächtig. Eine Lösung hängt davon ab
    welche Person geschickter mit einer Machtposition umgeht. Die anderen
    müssen eine Lösung schließlich akzeptieren. In solchen Teams sollte eine
    Schlichtung stattfinden um die Kooperationsbereitschaft zu erhöhen.
\end{itemize}
Zuletzt werden Biasfaktoren betrachtet:
\begin{itemize}
    \item Confirmation Bias - nur Informationen berücksichtigen zur
        Unterstützung einer vorgefassten Meinung (``abnicken''). Bejahen
        Teammitglieder potentiell falsche Antworten, ohne weiter darüber
        nachzudenken, spricht man von \textbf{compliance}. Beeinflusst
        die Gruppe die EInstellung des Menschen, so liegt
        \textbf{aaceptance} vor.
    \item Shared-Information-Bias - Es werden vornehmlich geteilte Informationen
        in den Entscheidungsprozess einbezogen, ohne private voll zu
        berücksichtigen. Zuträgliche Effekte sind:
        \begin{itemize}
            \item Negotiation-Bias - Argumente für geteilte Informationen
                weiter vorbringen und Argumente für private Information unter
                den Tisch fallen lassen.
            \item Group-Level-Discussion Bias - Über geteilte Informationen wird
                mehr gesprochen als pber private Informationen.
            \item Individual-Level-Evaluation Bias - Nur wirklich wichtige
                private Informationen kommen an die Bedeutung von geteilten
                Informationen ran.
        \end{itemize}
\end{itemize}
