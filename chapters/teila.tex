\subsection{Intuitives und Analytisches Entscheiden}
Es gibt zwei Arten von Entscheidungen:
\begin{itemize}
	\item System 1: intuitiv (Bauch), d.h. unbewusst, automatisch, extrem effizient,
	\item System 2: analytisch (Kopf), d.h. bewusst, langsam, aufwändig.
\end{itemize}
System 1 für einfache (d.h. die meisten), System 2 für schwere Entscheidung mit Unterstützung von System 1.\\
\textbf{Priming} bezeichnet die Beeinflussung der Verarbeitung eines Reizes durch Gedächtnisinhalte, die von einem vorangegangene Reiz aktiviert wurden.\\
\textbf{Belief-Bias} ist die Tendenz die Stärke von Argumenten (Prämissen) höher einzustufen um eine Aussage (Konklusion) zu bestätigen, nur weil diese scheinbar richtig ist.\\
\textbf{Verzerrende Faktoren} sind innere und äußere Faktoren, die eine Entscheidung beeinflussen (weg von der Rationalität).
\ \\
Eine Entscheidung ist in der Regeln nicht rein intuitiv oder rein analytisch, sondern
\begin{itemize}
	\item reflektiert-intuitiv, d.h. ein systematisch (teilanalytisch) mit Beobachtung der eigenen Intuition,
	\item rational, d.h. der Entscheidungsprozess umfasst Zielidentifizierung, angemessener Informationsbeschaffungsaufwand, Vermeidung von Verzerrung, Auswahl der nutzenmaximalen Alternative.
\end{itemize}
Bemerkung: Ein absolut rational entscheidener "Homo oeconomicus" ist in der Entscheidungstheorie nicht sinnvoll einsetzbar.\\
Ob eine Entscheidung rational oder reflektiert-intuitiv getroffen werden kann hängt von verschiedenen Faktoren ab:
\begin{itemize}
	\item Persönlichkeit des Entscheiders,
	\item Erfahrungswissen,
	\item Komplexität der Fragestellung,
	\item Zeit und Ressourcen.
\end{itemize}
Dazu folgende Begriffe und Überlegungen:\\
\textbf{Need for Cognition} Maß, welches beschreibt, wie viel Freude der Entscheider an kognitiv anspruchsvollen Aufgaben hat. Ein hohes Maß an NFC spricht für einen analytischen Ansatz der Entscheidungsfindung.\\
Hat der Entscheider viel Erfahrungswissen, kann er ggf. mehr intuitiv entscheiden. Gibt es kein Erfahrungswissen, so sind intuitive ENtscheidungen i.d.R. nicht sinnvoll. Bemerke Erfahrungswissen \(\neq\) Fachwissen.\\
Analytische Entscheidungen sind nur dann möglich, wenn die Fragestellung dies (ggf. mit Werkzeugen wie Computern) zulässt. Ist die Fragestellung zu komplex, gibt es womöglich keine analytischen Verfahren, welches nicht fehleranfällig ist. Weniger komplexe Fragestellungen lassen sich stets analytisch oder intuitiv lösen.\\
Der Aufwand (d.h. z.B.  Zeit, Geld), welcher zur Entscheidungsfindung investiert wird sollte in einem angemessenen Verhältnis zur Wichtigkeit oder Relevanz der Entscheidung stehen.\\

\subsection{Grundlegende Motive, Werte und Ziele von Menschen}
Jeder Mensch trifft seine Entscheidungen nach grundlegenden Motiven und Bedürfnissen. Diese werden (klassisch) in der Bedürfnishierarchie (Bedürfnispyramide) nach \textsc{Maslow} dargestellt. Die größte Relevanz in der Entscheidungslehre haben folgende Bedürfnisse:
\begin{itemize}
	\item Sicherheit (Risikoaversion),
	\item Kontakt (Personaleinsatzplanung),
	\item Kognitive Bedürfnisse (Kontrolle),
	\item Selbstwert (Dissonanztheorie).
\end{itemize}
Zu den Begriffen in Klammern gibt es in Teil B und C Erläuterungen.\\
Ein weiteres Modell stammt von \textsc{Reiss}. Er formulierte ein lineares System (d.h. nicht hierarchisch) mit 16 Lebensmotiven (z.B. Macht, Ordnung, Ehre, Familie), welche 1. temorär sind, d.h. sie müssen ständig wieder erfüllt sein, 2. individuelle gewichtet sind, also für verschiedene Menschen unterschiedlich bedeutsam sind. \textsc{Reiss} unterscheidet ausßerdem zwischen Instrumental- und Fundamentalmotiven -- das Verständnis ist ähnlich wie in der Entscheidungstheorie (auch dazu später mehr).\\
Nun betrachten wir das Kulturmodell nach \textsc{Schwartz}. Hier werden 57 Basiswerte (z.B. Hierarchie, Herrschaft, Harmonie) hinsichtlich ihrer relativen Bedeutung in 93 Ethnogruppen verglichen und sortiert. So bekommt der Basiswert (intellektuelle Autonomie) in Afrika den Wert 10 (niedrigster Wert), in Westeuropa den Wert 90 (höchster Wert).\\
\textsc{Hofstede} befragte von 1967 bis 1973 \(10^5\) Mitarbeiter von IBM aus 72 Nationen, um den Erfolg verschiedener Managmentpraktiken in Abhängigkeit von der jeweiligen Kultur zu analysieren. Also ein unternehmerischer Ansatz, als oben. \textsc{Hofstede} vergleicht dabei fünf "Kulturdimensionen":
\begin{itemize}
	\item Machtdistanz: Ausprägung des Respekts gegenüber Autoritäten in der Gesellschaft, wie weit wird akzeptiert, dass Menschen hinsichtlich ihrer Fähigkeiten unterschiedlich sind?
	\item Individualismus: Haben Menschen primär ihre eigenen interessen im Blick oder verfolgen sie eine kollektivistische Sichtweise?
	\item Maskulinität: Gibt es Rollenunterschiede der Geschlechter in der Gesellschaft und herrschen eher maskuline Werte (Geld, Macht) oder eher feminine Werte (Gesundheit, Fürsorge)?
	\item Unsicherheitsvermeidung: Werden Zukunfstunsicherheiten mit kontrollierter Risikobereitschaft akzeptiert oder wird eher Sicherheit angestrebt?
	\item Langzeitorientierung: Wird die Betrachtungsweise eher auf den kurz- oder langfristigen Erfolg der Gesellschaft gelegt?
\end{itemize}
Abschließend zu diesen Modellen sei erwähnt, dass sich die grundlegenden Werte zwischen den Generationen stets im Wandel befinden. So ist in der "Babyboomer"-Generationen u.a. eine stark pazifistische Einstellung vorzufinden (68er), während in der "Generation Y" die ökologische Verantwortung wichtiger ist.\\
\ \\
Eine Entscheidungssituation kann in verschiedenen Kontexten gesehen werden:
\begin{itemize}
	\item Soziale Norm: Wunsch nach Kontakt, Zugehörigkeit, Soziale Wärme, Freunde und Familie,
	\item Marktnorm: Geld, Leistung/Gegenleistung, Karriere, Selbstbestätigung, Individualismus.
\end{itemize}
Entscheidungen je nach Norm unterschiedlich getroffen, Menschen tendieren zum Beispiel eher dazu eine Leistung umsonst zu erbringen (etwas gutes tun, sozial), als nur vergünstigt (Leistung/Gegenleistung, Markt).\\
\ \\
Als nächstes untersuchen wir den Fairnessbegriff, dzu definieren wir:\\
\textbf{Ultimatumspiel}: Spieler A mit Geldbetrag 100€ entscheidet "Wie viel gebe ich Spieler B ab?" Spieler B entscheidet "Nehme ich das Angebot an?" ("Empfinde ich das Angebot als fair?") -- Falls nein gehen beide Spieler leer aus.\\
\textbf{Diktatorspiel}: Spieler A mit Geldbetrag 100€ entscheidet "Wie viel gebe ich Spieler B ab?" Spieler B hat keine Wahl.\\
Echte Fairnessüberlegung Spieler A: "Welche Aufteilung ist fair?" (in beiden Spielen).\\
Homo oeconomicus-Überlegung Spieler A: "Wie hoch sind die Annahmewahrscheinlichkeiten von Spieler B in Abhängigkeit vom angebotenen Betrag?" (nur im Ultimatumspiel)\\
Empirisch: Spieler A gibt im Diktatorspiel deutlich weniger ab als im Ultimatumspiel (keine echte Fairness). Dies wird verstärkt dadurch, wenn weder xperimentleiter noch der Spielpartner sieht, wer das Angebot macht.\\
Fazit: Menschen sind nicht immer fair, aber wollen stets als fair wahrgenommen werden. Im anonymen Kontext kommen weniger soziale Präferenzen zum Tragen, bei geringer sozialer Distanz dafür mehr.

\subsection{Wirkungsmodell und Umgang mit Wahrscheinlichkeiten}
Ein \textbf{Ziel} beschreibt einen bewertungsrelevanten Aspekt, ohne Angabe einer konkreten Ausprägung, lediglich mit einer Richtungsangabe.\\
Ein Ziel ist somit nicht etwa ein "erstrebenswerter Zustand".\\
In der präskriptiven Entscheidungstheorie (Teil B) sind Ziele (Zielskalen) die ersten zu findenen Faktoren eines Entscheidungsproblems, die später in einer optimalen Alternative des Wirkunsgmodell (gleich definiert) möglichst gut ausgeprägt sein sollen.\\
Es wird unterschieden in
\begin{itemize}
	\item Fundamentalzielen, besitzen einen Wert für sich und
	\item Instrumentalzielen, diese sind nur "Mittel zum Zweck".
\end{itemize}
Ziele sollten für das Wirkungsmodell möglichst fundamental formuliert sein. Gewichtet bzw. berücksichtigt man Instrumentalziele erhält man womöglich kein wirklich optimales Ergebnis.\\
\ \\
Viele Entscheidungen betreffen nicht die eigene Person oder werden nicht allein getroffen. Die Art wie man Ziele Dritter berücksichtigt im eigenen Zielsystem sind unterschiedlich:
\begin{itemize}
	\item gezwungen, Loyalität, Hierarchische Beziehung (Unterscheide: konkrete Weisung, keine konkrete Weisung),
	\item freiwillig, Altruismus, Altruistische Beziehung.
\end{itemize}

% eof
