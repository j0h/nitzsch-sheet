\subsection{Intuitives und Analytisches Entscheiden}
Es gibt zwei Arten von Entscheidungen:
\begin{itemize}
	\item System 1: intuitiv (Bauch), d.h. unbewusst, automatisch, extrem effizient,
	\item System 2: analytisch (Kopf), d.h. bewusst, langsam, aufwändig.
\end{itemize}
System 1 für einfache (d.h. die meisten), System 2 für schwere Entscheidung mit Unterstützung von System 1.\\
\textbf{Priming} bezeichnet die Beeinflussung der Verarbeitung eines Reizes durch Gedächtnisinhalte, die von einem vorangegangene Reiz aktiviert wurden.\\
\textbf{Belief-Bias} ist die Tendenz die Stärke von Argumenten (Prämissen) höher einzustufen um eine Aussage (Konklusion) zu bestätigen, nur weil diese scheinbar richtig ist.\\
\textbf{Verzerrende Faktoren} sind innere und äußere Faktoren, die eine Entscheidung beeinflussen (weg von der Rationalität).
\ \\
Eine Entscheidung ist in der Regeln nicht rein intuitiv oder rein analytisch, sondern
\begin{itemize}
	\item reflektiert-intuitiv, d.h. ein systematisch (teilanalytisch) mit Beobachtung der eigenen Intuition,
	\item rational, d.h. der Entscheidungsprozess umfasst Zielidentifizierung, angemessener Informationsbeschaffungsaufwand, Vermeidung von Verzerrung, Auswahl der nutzenmaximalen Alternative.
\end{itemize}
Bemerkung: Ein absolut rational entscheidener "Homo oeconomicus" ist in der Entscheidungstheorie nicht sinnvoll einsetzbar.\\
Ob eine Entscheidung rational oder reflektiert-intuitiv getroffen werden kann hängt von verschiedenen Faktoren ab:
\begin{itemize}
	\item Persönlichkeit des Entscheiders,
	\item Erfahrungswissen,
	\item Komplexität der Fragestellung,
	\item Zeit und Ressourcen.
\end{itemize}
Dazu folgende Begriffe und Überlegungen:\\
\textbf{Need for Cognition} Maß, welches beschreibt, wie viel Freude der Entscheider an kognitiv anspruchsvollen Aufgaben hat. Ein hohes Maß an NFC spricht für einen analytischen Ansatz der Entscheidungsfindung.\\
Hat der Entscheider viel Erfahrungswissen, kann er ggf. mehr intuitiv entscheiden. Gibt es kein Erfahrungswissen, so sind intuitive ENtscheidungen i.d.R. nicht sinnvoll. Bemerke Erfahrungswissen \(\neq\) Fachwissen.\\
Analytische Entscheidungen sind nur dann möglich, wenn die Fragestellung dies (ggf. mit Werkzeugen wie Computern) zulässt. Ist die Fragestellung zu komplex, gibt es womöglich keine analytischen Verfahren, welches nicht fehleranfällig ist. Weniger komplexe Fragestellungen lassen sich stets analytisch oder intuitiv lösen.\\
Der Aufwand (d.h. z.B.  Zeit, Geld), welcher zur Entscheidungsfindung investiert wird sollte in einem angemessenen Verhältnis zur Wichtigkeit oder Relevanz der Entscheidung stehen.\\

\subsection{Grundlegende Motive, Werte und Ziele von Menschen}


\subsection{Wirkungsmodell und Umgang mit Wahrscheinlichkeiten}

% eof
