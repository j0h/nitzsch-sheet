Die \textbf{deskriptive Entscheidungstheorie} beschäftigt sich mit
\textbf{Erkenntnissen, die man über das menschliche Entscheidungsverhalten} hat.

\subsection{Kognitive Ursachen für unvollkommene Informationsverarbeitung}
Man kann Informationsverarbeitung in 3 Phasen gliedern:
\begin{itemize}
    \item Wahrnehmung - vornehmlich visuell oder auditiv

    \item Verarbeitung - durch Hirn, räumlich oder verbal
    \item Reaktion - manuell oder verbal
\end{itemize}
Es kommen bei der \textbf{Wahrnehmung} mehrere Faktoren ins Spiel:
\begin{itemize}
    \item Aufmerksamkeit
    \begin{itemize}
        \item streng limitiert
        \item System 1 (spontan, schonend, ungenau, unterstützend),
            System 2 (ressourcenfresser)

    \end{itemize}
    \item Wahrnehmung und Verfügbarkeit
    \begin{itemize}
        \item Vereinfachung - z.B. B besser A, C besser A aber A besser C.
            Nicht rational
        \item Selektive Wahrnehmung - wahrnehmen, was man will.
        \begin{itemize}
            \item \textbf{Confirmation Bias} - nur nach meinungskonformen
            Umwelteinflüssen filtern (z.B. nur Vorteile suchen nach Autokauf
            um Kauf zu bestätigen).
            \item \textbf{Spreading-Apart-Effekt} - bestellte Automarke
            in Werbung hängen bleiben, nicht so bei verworfenen
            Entscheidungsalternativen.
        \end{itemize}
        \item Kontrasteffekte - in kontrast stehende Informationen werden
            überhöht wahrgenommen. (z.B. zwei gleiche Kugeln im Kontrast
            zu größeren/kleineren Kugeln wirken unterschiedlich groß)
    \end{itemize}
\end{itemize}
