Die \textbf{deskriptive Entscheidungstheorie} beschäftigt sich mit
\textbf{Erkenntnissen, die man über das menschliche Entscheidungsverhalten} hat.

\subsection{Kognitive Ursachen für unvollkommene Informationsverarbeitung}
Man kann Informationsverarbeitung in 3 Phasen gliedern:
\begin{itemize}
    \item Wahrnehmung - vornehmlich visuell oder auditiv

    \item Verarbeitung - durch Hirn, räumlich oder verbal
    \item Reaktion - manuell oder verbal
\end{itemize}
Es kommen bei der \textbf{Wahrnehmung} mehrere Faktoren ins Spiel:
\begin{itemize}
    \item Aufmerksamkeit
    \begin{itemize}
        \item streng limitiert
        \item System 1 (spontan, schonend, ungenau, unterstützend),
            System 2 (ressourcenfresser)

    \end{itemize}
    \item Wahrnehmung und Verfügbarkeit
    \begin{itemize}
        \item Vereinfachung - z.B. B besser A, C besser A aber A besser C.
            Nicht rational
        \item Selektive Wahrnehmung - wahrnehmen, was man will.
        \begin{itemize}
            \item \textbf{Confirmation Bias} - nur nach meinungskonformen
            Umwelteinflüssen filtern (z.B. nur Vorteile suchen nach Autokauf
            um Kauf zu bestätigen).
            \item \textbf{Spreading-Apart-Effekt} - bestellte Automarke
            in Werbung hängen bleiben, nicht so bei verworfenen
            Entscheidungsalternativen.
        \end{itemize}
        \item Kontrasteffekte - in kontrast stehende Informationen werden
            überhöht wahrgenommen. (z.B. zwei gleiche Kugeln im Kontrast
            zu größeren/kleineren Kugeln wirken unterschiedlich groß)
    \end{itemize}
\end{itemize}
Bei der Verfügbarkeit von Gedächtnisinhalten wird zwischen \textbf{Lang-}
und \textbf{Kurzzeitgedächtnis} unterschieden. Der \textbf{primacy-Effekt}
beschreibt die Überführung ins Langzeitgedächtnis während der
\textbf{recency-Effekt} die (kurzzeitige) Abrufbarkeit frischen Wissens aus dem
Kurzzeitgedächtnis beschreibt.
\textbf{Chunking} beschreibt die Aufteilung von Informationen zur besseren
Aufnahme in das Kurzzeitgedächtnis (Kompression der Inhalte). Das
Langzeitgedächtnis hingegen bildet als Netzwerk aus Assoziationen Daten ab.
So werden semantisch verwandte Informationen miteinander verknüpft. Die
Verfügbarkeit und damit die Abrufbarkeit von Daten im Langzeitgedächtnis
hängt von verschiedenen Faktoren ab:
\begin{itemize}
    \item Aktualität
    \item Anschaulichkeit - einfache vs abstrakte Informationen
    \item Auffälligkeit
    \item Aufmerksamkeit bei Aufnahme der Information
    \item Abruffrequenz
\end{itemize}

\subsection{Narrow Thinking und Heuristiken}
