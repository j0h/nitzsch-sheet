Die \textbf{deskriptive Entscheidungstheorie} beschäftigt sich mit
\textbf{Erkenntnissen, die man über das menschliche Entscheidungsverhalten} hat.

\subsection{Kognitive Ursachen für unvollkommene Informationsverarbeitung}
Man kann Informationsverarbeitung in 3 Phasen gliedern:
\begin{itemize}
    \item Wahrnehmung - vornehmlich visuell oder auditiv

    \item Verarbeitung - durch Hirn, räumlich oder verbal
    \item Reaktion - manuell oder verbal
\end{itemize}
Es kommen bei der \textbf{Wahrnehmung} mehrere Faktoren ins Spiel:
\begin{itemize}
    \item Aufmerksamkeit
    \begin{itemize}
        \item streng limitiert
        \item System 1 (spontan, schonend, ungenau, unterstützend),
            System 2 (ressourcenfresser)

    \end{itemize}
    \item Wahrnehmung und Verfügbarkeit
    \begin{itemize}
        \item Vereinfachung - z.B. B besser A, C besser A aber A besser C.
            Nicht rational
        \item Selektive Wahrnehmung - wahrnehmen, was man will.
        \begin{itemize}
            \item \textbf{Confirmation Bias} - nur nach meinungskonformen
            Umwelteinflüssen filtern (z.B. nur Vorteile suchen nach Autokauf
            um Kauf zu bestätigen).
            \item \textbf{Spreading-Apart-Effekt} - bestellte Automarke
            in Werbung hängen bleiben, nicht so bei verworfenen
            Entscheidungsalternativen.
        \end{itemize}
        \item Kontrasteffekte - in kontrast stehende Informationen werden
            überhöht wahrgenommen. (z.B. zwei gleiche Kugeln im Kontrast
            zu größeren/kleineren Kugeln wirken unterschiedlich groß)
    \end{itemize}
\end{itemize}
Bei der Verfügbarkeit von Gedächtnisinhalten wird zwischen \textbf{Lang-}
und \textbf{Kurzzeitgedächtnis} unterschieden. Der \textbf{primacy-Effekt}
beschreibt die Überführung ins Langzeitgedächtnis während der
\textbf{recency-Effekt} die (kurzzeitige) Abrufbarkeit frischen Wissens aus dem
Kurzzeitgedächtnis beschreibt.
\textbf{Chunking} beschreibt die Aufteilung von Informationen zur besseren
Aufnahme in das Kurzzeitgedächtnis (Kompression der Inhalte). Das
Langzeitgedächtnis hingegen bildet als Netzwerk aus Assoziationen Daten ab.
So werden semantisch verwandte Informationen miteinander verknüpft. Die
Verfügbarkeit und damit die Abrufbarkeit von Daten im Langzeitgedächtnis
hängt von verschiedenen Faktoren ab:
\begin{itemize}
    \item Aktualität
    \item Anschaulichkeit - einfache vs abstrakte Informationen
    \item Auffälligkeit
    \item Aufmerksamkeit bei Aufnahme der Information
    \item Abruffrequenz
\end{itemize}

\subsection{Narrow Thinking und Heuristiken}
\textbf{Narrow Thinking} beschreibt eine Menge von Effekten, die sich damit
beschäftigen, dass der Mensch sich gedanklich möglichst ressourcensparend
- also in einem engen Umfeld - bewegt.
Zu diesem Effekten gehören:
\begin{itemize}
    \item Verfügbarkeit von Informationen
    \item mentale Kontenführung
    \item Over-Confidence
    \item unbewusste Heurisiken
\end{itemize}
Verfügbarkeitseffekte lassen sich wie folgt gliedern:
\begin{itemize}
    \item direkter Einfluss
    \begin{itemize}
        \item \textbf{Overreaction} - werden nicht alle wichtigen Aspekte in eine
            Entscheidung einbezogen, kommt es zu Verzerrungen in den
            Sachverhalten. Somit beeinflussen aktuelle, präsente Informationen
            stärker als weniger aktive Informationen. Es kann zu Überreaktionen
            kommen (z.B. 9/11 - Einbruch Börsenkurs).
        \item \textbf{Narrative Bias} - verstärkte Gewichtung von
            vereinfachten Fakten, die z.B. durch Geschichten anstatt
            abstrakt vermittelt wurden.
        \item \textbf{Primacy Effekt} - zuerst gelesener Chunk am
            präsentesten da dieser konzentriert aufgenommen wurde und
            auf dem Weg ins Langzeitgedächtnis ist. $\rightarrow$ mit
            positiver Nachricht anfangen wenn negative Nachricht besser
            aufgenommen werden soll.
    \end{itemize}
    \item indirekter Einfluss
    \begin{itemize}
        \item \textbf{Priming Effekt} - unbewusste Beeinflussung durch Umwelteinflüsse.
            Zum Beispiel zuvor gelesene Worte oder gesehen Bilder.
    \end{itemize}
\end{itemize}
Mentale Konten (\textbf{mental accounting}), sind Konstrukte, in denen Informationen verbucht werden.
Informationen, die zum Beispiel zu einem Projekt gehören werden in
einem mentalen Konto verbucht, ohne Wechselwirkungen zu anderen Projekten
zu beachten (Gedankenexperiment verlorene Kontertkarte vs. Verlust 100EUR). \\
\textbf{Verankerungsheuristiken} beschreiben den Effekt, dass Menschen sich
an einem Ausgangswert (\textbf{Anchoring}) orientieren und sich dann durch weitere Analyse einem
besseren Wert annähern (\textbf{Adjustment}). Hieraus resultieren
einige Verzerrungen. Nämlich der \textbf{Status-Quo Bias}, der beschreibt am bestehenden
festzuhalten. Bei der Schätzung zusammengesetzer Wahrscheinlichkeiten
kommt es sogar zu gravierenden Unter- oder Überschätzungen wenn der Anker fern
des realistischen Wertes liegt. \\
\textbf{Repräsentativitätsheuristiken} beschreiben Effekte, die durch den
Wegfall der Analyse von Informationen wegfallen. Passen Informationen
in ein Schema, werden diese diesem zugerechnet, ohne eben jene Information
gesondert abzuspeichern. Daraus resultiert, dass bei Wahrscheinlichkeiten
große Äquivalenzklassen von Ereignissen eine intuitiv höhere
Wahrscheinlichkeit besitzen (23415 vs 66666, genannt \textbf{Gamblers
Fallacy}). Bei zusammengesetzen
Ereignissen kann es vorkommen, dass bedingte Wahrscheinlichkeiten nicht
beachtet werden und Ereignisse bedeutend zu hoch bewertet werden
(\textbf{Conjunction Fallacy}). cite: Plausibilität ist einer der
größten Feinde der Wahrheit. Entsprechend werden beobachtete Zusammenhänge
vorschnell in ein Schema gepresst. \\

\textbf{Overconfidence} beschreibt das Phänomen der Selbstüberschätzung
und lässt sich in 3 Varianten unterteilen:
\begin{itemize}
    \item Overestimation - Überschätzung (besser als fixer Wert)
    \item Overplacement - Überschätzung im Kontext (better than average).
    \item Miscalibration - Überschätzung eines Konfidenzintervalls
        (unter den 90\% besten).
\end{itemize}
In diesem Zusammenhang ist WYSIATI (what you see is all there is) wichtig.
Es beschreibt den Zusammenhang, dass Erkenntnisse aus einem subjektiven
Kontext abgeleitet werden.\\
\textbf{Regressivität} beschreibt die Verzerrung einer Selbsteinschätzung
zu einem Mittelwert. Dies ist eine Folge einer gewissen Unsicherheit.
Bei leichten Aufgaben kommt es zu einer Underestamination, während schwierige
Aufgaben eher eine Overestamination zur Folge haben.

\subsection{Rationalitätsgefährdende Motive der Menschen}
Der Mensch dürstet nach gewissen, für Entscheidungen gefährlichen,
Grundbedürfnissen - hoher Selbstwert, Dissonanzfreiheit, Kontrolle.
Betrachtet man Selbstwert, so kommt die \textbf{Attributionstheorie}
ins Spiel, die sich mit der Zuweisung von Erfolgen und Misserfolgen
beschäftigt (schlechter Ausgang - Pech oder Unfähigkeit?).
\begin{itemize}
    \item situative Attribution - Rückführung auf Pech
    \item dispositionale Attribution - Selbstbezug
\end{itemize}
Auf positive Ereignisse folgt eher dispositionale Attribution, also eine
\textbf{selbstwertdienliche Attribution}, während bei negativen
Ereignissen eine situative Attribution stattfindet. Dies folgt direkt
aus dem Bedürfnis nach hohem Selbstwert.
