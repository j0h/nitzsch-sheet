Die \textbf{deskriptive Entscheidungstheorie} beschäftigt sich mit
\textbf{Erkenntnissen, die man über das menschliche Entscheidungsverhalten} hat.

\subsection{Kognitive Ursachen für unvollkommene Informationsverarbeitung}
Man kann Informationsverarbeitung in 3 Phasen gliedern:
\begin{itemize}
    \item Wahrnehmung - vornehmlich visuell oder auditiv

    \item Verarbeitung - durch Hirn, räumlich oder verbal
    \item Reaktion - manuell oder verbal
\end{itemize}
Es kommen bei der \textbf{Wahrnehmung} mehrere Faktoren ins Spiel:
\begin{itemize}
    \item Aufmerksamkeit
    \begin{itemize}
        \item streng limitiert
        \item System 1 (spontan, schonend, ungenau, unterstützend),
            System 2 (ressourcenfresser)

    \end{itemize}
    \item Wahrnehmung und Verfügbarkeit
    \begin{itemize}
        \item Vereinfachung - z.B. B besser A, C besser A aber A besser C.
            Nicht rational
        \item Selektive Wahrnehmung - wahrnehmen, was man will.
        \begin{itemize}
            \item \textbf{Confirmation Bias} - nur nach meinungskonformen
            Umwelteinflüssen filtern (z.B. nur Vorteile suchen nach Autokauf
            um Kauf zu bestätigen).
            \item \textbf{Spreading-Apart-Effekt} - bestellte Automarke
            in Werbung hängen bleiben, nicht so bei verworfenen
            Entscheidungsalternativen.
        \end{itemize}
        \item Kontrasteffekte - in kontrast stehende Informationen werden
            überhöht wahrgenommen. (z.B. zwei gleiche Kugeln im Kontrast
            zu größeren/kleineren Kugeln wirken unterschiedlich groß)
    \end{itemize}
\end{itemize}
Bei der Verfügbarkeit von Gedächtnisinhalten wird zwischen \textbf{Lang-}
und \textbf{Kurzzeitgedächtnis} unterschieden. Der \textbf{primacy-Effekt}
beschreibt die Überführung ins Langzeitgedächtnis während der
\textbf{recency-Effekt} die (kurzzeitige) Abrufbarkeit frischen Wissens aus dem
Kurzzeitgedächtnis beschreibt.
\textbf{Chunking} beschreibt die Aufteilung von Informationen zur besseren
Aufnahme in das Kurzzeitgedächtnis (Kompression der Inhalte). Das
Langzeitgedächtnis hingegen bildet als Netzwerk aus Assoziationen Daten ab.
So werden semantisch verwandte Informationen miteinander verknüpft. Die
Verfügbarkeit und damit die Abrufbarkeit von Daten im Langzeitgedächtnis
hängt von verschiedenen Faktoren ab:
\begin{itemize}
    \item Aktualität
    \item Anschaulichkeit - einfache vs abstrakte Informationen
    \item Auffälligkeit
    \item Aufmerksamkeit bei Aufnahme der Information
    \item Abruffrequenz
\end{itemize}

\subsection{Narrow Thinking und Heuristiken}
\textbf{Narrow Thinking} beschreibt eine Menge von Effekten, die sich damit
beschäftigen, dass der Mensch sich gedanklich möglichst ressourcensparend
- also in einem engen Umfeld - bewegt.
Zu diesem Effekten gehören:
\begin{itemize}
    \item Verfügbarkeit von Informationen
    \item mentale Kontenführung
    \item Over-Confidence
    \item unbewusste Heurisiken
\end{itemize}
Verfügbarkeitseffekte lassen sich wie folgt gliedern:
\begin{itemize}
    \item direkter Einfluss
    \begin{itemize}
        \item \textbf{Overreaction} - werden nicht alle wichtigen Aspekte in eine
            Entscheidung einbezogen, kommt es zu Verzerrungen in den
            Sachverhalten. Somit beeinflussen aktuelle, präsente Informationen
            stärker als weniger aktive Informationen. Es kann zu Überreaktionen
            kommen (z.B. 9/11 - Einbruch Börsenkurs).
        \item \textbf{Narrative Bias} - verstärkte Gewichtung von
            vereinfachten Fakten, die z.B. durch Geschichten anstatt
            abstrakt vermittelt wurden.
        \item \textbf{Primacy Effekt} - zuerst gelesener Chunk am
            präsentesten da dieser konzentriert aufgenommen wurde und
            auf dem Weg ins Langzeitgedächtnis ist. $\rightarrow$ mit
            positiver Nachricht anfangen wenn negative Nachricht besser
            aufgenommen werden soll.
    \end{itemize}
    \item indirekter Einfluss
    \begin{itemize}
        \item \textbf{Priming Effekt} - unbewusste Beeinflussung durch Umwelteinflüsse.
            Zum Beispiel zuvor gelesene Worte oder gesehen Bilder.
    \end{itemize}
\end{itemize}
Mentale Konten (\textbf{mental accounting}), sind Konstrukte, in denen Informationen verbucht werden.
Informationen, die zum Beispiel zu einem Projekt gehören werden in
einem mentalen Konto verbucht, ohne Wechselwirkungen zu anderen Projekten
zu beachten (Gedankenexperiment verlorene Kontertkarte vs. Verlust 100EUR). \\
\textbf{Verankerungsheuristiken} beschreiben den Effekt, dass Menschen sich
an einem Ausgangswert (\textbf{Anchoring}) orientieren und sich dann durch weitere Analyse einem
besseren Wert annähern (\textbf{Adjustment}). Hieraus resultieren
einige Verzerrungen. Nämlich der \textbf{Status-Quo Bias}, der beschreibt am bestehenden
festzuhalten. Bei der Schätzung zusammengesetzer Wahrscheinlichkeiten
kommt es sogar zu gravierenden Unter- oder Überschätzungen wenn der Anker fern
des realistischen Wertes liegt. \\
\textbf{Repräsentativitätsheuristiken} beschreiben Effekte, die durch den
Wegfall der Analyse von Informationen wegfallen. Passen Informationen
in ein Schema, werden diese diesem zugerechnet, ohne eben jene Information
gesondert abzuspeichern. Daraus resultiert, dass bei Wahrscheinlichkeiten
große Äquivalenzklassen von Ereignissen eine intuitiv höhere
Wahrscheinlichkeit besitzen (23415 vs 66666, genannt \textbf{Gamblers
Fallacy}). Bei zusammengesetzen
Ereignissen kann es vorkommen, dass bedingte Wahrscheinlichkeiten nicht
beachtet werden und Ereignisse bedeutend zu hoch bewertet werden
(\textbf{Conjunction Fallacy}). cite: Plausibilität ist einer der
größten Feinde der Wahrheit. Entsprechend werden beobachtete Zusammenhänge
vorschnell in ein Schema gepresst. \\
\textbf{Overconfidence} beschreibt das Phänomen der Selbstüberschätzung
und lässt sich in 3 Varianten unterteilen:
\begin{itemize}
    \item \textbf{Overestimation} - Überschätzung (besser als fixer Wert)
    \item \textbf{Overplacement} - Überschätzung im Kontext (better than average).
    \item \textbf{Miscalibration} - Überschätzung eines Konfidenzintervalls
        (unter den 90\% besten).
\end{itemize}
In diesem Zusammenhang ist WYSIATI (what you see is all there is) wichtig.
Es beschreibt den Zusammenhang, dass Erkenntnisse aus einem subjektiven
Kontext abgeleitet werden.\\
\textbf{Regressivität} beschreibt die Verzerrung einer Selbsteinschätzung
zu einem Mittelwert. Dies ist eine Folge einer gewissen Unsicherheit.
Bei leichten Aufgaben kommt es zu einer Underestamination, während schwierige
Aufgaben eher eine Overestamination zur Folge haben.

\subsection{Rationalitätsgefährdende Motive der Menschen}
Der Mensch dürstet nach gewissen, für Entscheidungen gefährliche,
Grundbedürfnissen - hoher Selbstwert, Dissonanzfreiheit, Kontrolle.
Betrachtet man Selbstwert, so kommt die \textbf{Attributionstheorie}
ins Spiel, die sich mit der Zuweisung von Erfolgen und Misserfolgen
beschäftigt (schlechter Ausgang - Pech oder Unfähigkeit?).
\begin{itemize}
    \item \textbf{situative Attribution} - Rückführung auf Pech
    \item \textbf{dispositionale Attribution} - Selbstbezug
\end{itemize}
Auf positive Ereignisse folgt eher dispositionale Attribution, also eine
\textbf{selbstwertdienliche Attribution}, während bei negativen
Ereignissen eine situative Attribution stattfindet. Dies folgt direkt
aus dem Bedürfnis nach hohem Selbstwert.
Betrachtet man das Bedürfnis nach Dissonanzfreiheit so betrachtet man
Bewusstseinsprozesse (\textbf{Kognitionen}). Die Dissonanzfreiheit
beschreibt den Wunsch nach einem konsistenten System von Meinungs- Glaubens und
Wissenseinheiten. \textbf{Hypothesen} sind subjektive Handlungs und
Erkenntnissentscheidungen, während Kognitionen externe Informationen
wiedergeben. Ein Zitat, also das aufgreifen einer Aussage ist somit eine
Kognition. Das Glauben der Aussage ist somit eine Hypothese. Kommt nun eine
widersprüchlige Kognition hinzu, entsteht eine \textbf{Inkonsistenz}.
Im allgemeinen strebt ein Individuum nach der \textbf{Vermeidung von
Inkonsistenzen}. Am einfachsten geschieht dies durch die Suche
nach Kognitionen, die konsistenz zum bestehenden System sind. Auflösungen
von Inkonsistenzen ist nur durch das Entfernen von Kognitionen möglich.
\textbf{Wichtig:} Dissonanzen treten nur in einem System auf, in dem mindestens
eine Hypothese existiert. \textbf{Ohne eine Entscheidung gibt es keine
Dissonanz}. In diesem Zusammenhang bietet ein \textbf{Commitment} eine
notwendige Bedingung für Dissonanz. Ein Commitment ist ein Zusammenhang,
der ausdrückt, dass man an einer Entscheidung emotional hängt. Die Ausprägung
des Commitments hängt von 4 Faktoren ab:
\begin{itemize}
    \item \textbf{Entscheidungsfreiheit} - impliziert Übernahme von Verantwortung
        und Wahlfreiheit zwischen Entscheidungen. Impliziert wiederum
        emotionale Abhängigkeit.
    \item \textbf{Verantwortung}
    \item \textbf{irreversible Kosten} - Revision der Entscheidung
        ist kostspielig. Höhere emotionale Abhängigkeit.
    \item \textbf{Normabweichung} - Hipster haben ein hohes Commitment gegnüber
        ihres Lebensstils da es als etwas ``außergewöhnliches'' gilt.
\end{itemize}

Zur Auflösung von Dissonanz kann entweder die Entscheidung revidiert werden,
die Entscheidung durch andere Entscheidungen zum Erfolg geführt werden
(\textbf{Sunk-Cost-Falle}) oder die Wahrnehmung beeinflusstwerden
( \textbf{Confirmation Bias}).
\textbf{Closed-Minded}-Personen streben dauerhaft nach \textbf{Konsonanz}.
Hier ist die Gefahr von Wahrnehmungsbeeinflussung besonders groß.
\textbf{Open-Minded}-Personen streben zwar nach Konsonanz, sind aber bereit
Dissonanzen auf dem Weg zur Konsonanz in Kauf zu nehmen. Die Sunk-Cost-Falle
ist hier durchaus problematisch. Die Personengruppe kommt aber durch
Reflexion zur Dissonanzauflösung durch Kognitions- oder Hypothesenrevision.

Zuletzt strebt der Mensch nach \textbf{Kontrolle} da er sich selbst als aktiver Veränderer seiner
Umwelt wahrnehmen möchte. Kommt es zum Kontrollverlust, können sich
Krankheitsbilder verschärfen und das Wohlbefinden drastisch verschlechtern.
Es gibt verschiedene Varianten der Kontrolle \textbf{Locus of Control} -
wo liegt die Kontrolle/Ausprägung der Kontrolle:
\begin{itemize}
    \item \textbf{Fähigkeit zur Beeinflussung} - stärkste Form
    \item \textbf{Fähigkeit zur Vorhersage} - zukünftige Entscheidungen sind sicher durch
        Vorhersage statt durch Beeinflussung. Entscheider sieht seine Alternativen
        als fest an.
    \item \textbf{Kenntnis der Einflussvariablen in einer Entscheidungssituation}
        - Person unter Unsicherheit, lediglich mit der Möglichkeit durch
        Einschätzung der Einflussvariablen die eigene Situation einzuschätzen.
        Es ist beispielsweise möglich ein Risiko einzuschätzen.
    \item \textbf{Fähigkeit des retrospektiven Erklärens von Ereignissen} -
        Ist es möglich aus alten Ereignissen potentiell ohne Kontrolle noch
        Erfahrungswerte abzuleiten, dann ist ein solches Eregnis für
        zukünftige Kontroll und Entscheidungssituationen wertvoll.
\end{itemize}

Im Falle der Kenntniss der Einflussvariablen sind drei Variablen
im Kontext der Finanzmärkte besonders wichtig:
\begin{itemize}
    \item \textbf{Höhe und Vorzeichen der Beträge} - niedrige Beträge ohne Zwang und mit
        Spielcharakter. Bei hohen negativen Beträgen besteht ein hoher Wunsch
        nach Kontrolle. Ein Abschluss einer Versicherung mit ausreichendem
        Deckungsbeitrag kann Kontrollbedürfnis stillen.
    \item \textbf{Ambiguität und Kompetenz} -
        Hohe Unsicherheit gibt Gefühl von Kontrollverlust.
    \item \textbf{Integration und Segregation von Mental Accounting} - Mehrere
        Ausspielungen können Risiken kompensieren (Gesetz der großen Zahlen).
        Wir dies in einem einzigen Mental Account verbucht
        (\textbf{Integration}), gelangt der Entscheider
        zu einem Kontrollgefühl. Werden viele Mental Accounts genutzt
        (\textbf{Segregation}), kommt es zu einem Kontrolldefizit.
\end{itemize}

Zuletzt lassen sich Verhaltensverzerrungen aus den Konsequenzen
der Vorüberlegungen ableiten:
\begin{itemize}
    \item \textbf{Unterlassung von Aktionen mit geringer Kontrollwahrnehmung}
        - Menschen meiden Situationen ohne Kontrolle. Das Kontrolldefizit ist
        am höchsten wenn die Sitaution ungewohnt ist (Personen, die zum ersten
        mal fliegen). Um das Kontrollgefühl unter vielen Alternativen zu erhalten
        sollten offensichtlich schlechte Alternativen sofort aussortiert werden.
        Dieses Problem nennt sich \textbf{Choice-Overload}. Unerwünscht sind
        Aktionen, die nur zum Schein eine bessere Kontrolle mit sich bringen.
        Beispielsweise bezeichnet das \textbf{Home Bias} den Effekt, dass
        Anleger nur inländische Aktien kaufen da es Ihnen den Anschein von Kontrolle
        gibt.
    \item \textbf{Kontrollillusion} - Hier handelt es sich um das Phänomen, das
        nur eingebildete Kontrolle beschreibt obwohl keine Kontrolle
        vorhanden ist. Kontrollillusion spielt somit auch in
        Overconfidence hinein. Der Entscheider bildet sich Sicherheit und damit
        Kontrolle ein. Im Falle der Reflexion einer Entscheidung, kann ein
        Entscheider den Ereignisausgang falsch für zukünftige Ereignisse
        einschätzen und so die eigene Fähigkeit vergangene Ereignisse einzuschätzen
        falsch einschätzen. Dieser Effekt heißt \textbf{Hindsight Bias}.
        Kontrollillusion ist selbstwertdienlich, was eine solche Situation
        fördert.
    \item \textbf{Stress- und Kontrollverlust-Phänomene}
        - \textbf{Illusion of Validity} - `` so viele Leute könne sich
        nicht irren.'' um wieder an Kontrolle zu gelangen.
        Nach einer Erfolgssträhne kann es zum Kontrollverlust kommen da
        neue Aufträge nicht mehr Reibungslos verlaufen, ohne dass die Gründe
        reflektiert werden können.
\end{itemize}
